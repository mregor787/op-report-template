\section{Место и сроки проведения практики}

\textit{
Cроки проведения практики
}

\textit{
-дата начала практики \underline{17.09.2021}
}

\textit{
-дата окончания практики \underline{12.07.2022}
}

\vspace{10pt}

\textit{
Наименование организации:
}

\textit{
\underline{Московский авиационный институт (национальный исследовательский универститет)}
}

\vspace{10pt}

\textit{
Название структурного подразделения (отдел, лаборатория):
}

\textit{
\underline{кафедра №806 \enquote{Вычислительная математика и программирование}}
}

\section{Инструктаж по технике безопасности}

\tline{(подпись проводившего)}{2in} / \underline{Крылов С. С.} / $\underset{\text{(дата проведения)}}{\uline{\text{\hspace{10pt}17 сентября\hspace{10pt}}}}$ 2021\,г.

\section{Индивидуальное задание обучающегося}

Принять участие в тренировках и соревнованиях по олимпиадному программированию для студентов первого курса в 2021/2022 учебном году: посетить и проработать установочные лекции, решать и дорешивать конкурсные задания, принять участие в разборе. Объём практики 108 часов.

\vspace{10pt}

\textit{Руководитель практики от МАИ:}

\underline{Крылов С. С.} / \underline{\hspace{100pt}} / \underline{17 сентября} 2021\,г.

\vspace{10pt}

\textit{Руководитель от организации:}

\underline{\hspace{100pt}} / \underline{\hspace{100pt}} / \underline{17 сентября} 2021\,г.

\vspace{20pt}

\tline{(подпись обучающегося)}{2in} / \underline{Иванов И. И.} / $\underset{\text{(дата)}}{\uline{\text{17 сентября}}}$ 2021\,г.

\pagebreak
