\section{Место и сроки проведения практики}

\textit{
Cроки проведения практики
}

\textit{
-дата начала практики \underline{29.06.2022}
}

\textit{
-дата окончания практики \underline{12.07.2022}
}

\vspace{10pt}

\textit{
Наименование организации:
}

\textit{
\underline{Московский авиационный институт (национальный исследовательский универститет)}
}

\vspace{10pt}

\textit{
Название структурного подразделения (отдел, лаборатория):
}

\textit{
\underline{кафедра №806 \enquote{Вычислительная математика и программирование}}
}

\section{Инструктаж по технике безопасности}

\tline{(подпись проводившего)}{2in} / \underline{Крылов С. С.} / $\underset{\text{(дата проведения)}}{\uline{\text{\hspace{10pt}29 июня\hspace{10pt}}}}$ 2022\,г.

\section{Индивидуальное задание обучающегося}

Принять участие в учебно-тренировочных контестах по олимпиадному программированию для студентов первого курса в течении 9 дней: посетить и проработать установочные лекции, решать и дорешивать конкурсные задания, принять участие в разборах контестов. Составить отчёт в форме журнала установленной формы и пройти процедуру защиты практики.

Объём практики 108 часов в течение 12 учебных дней.

\vspace{10pt}

\textit{Руководитель практики от МАИ:}

\underline{Крылов С. С.} / \underline{\hspace{100pt}} / \underline{29 июня} 2022\,г.

\vspace{10pt}

\textit{Руководитель от организации:}

\underline{\hspace{100pt}} / \underline{\hspace{100pt}} / \underline{29 июня} 2022\,г.

\vspace{20pt}

\tline{(подпись обучающегося)}{2in} / \underline{Иванов И. И.} / $\underset{\text{(дата)}}{\uline{\text{29 июня}}}$ 2022\,г.

\pagebreak
